\documentclass[aspectratio=169]{beamer}

\usepackage{./mpidrtheme/mpidr21}

%%this package is only here to include latex-sourcecode in the example-presentation
\usepackage{verbatim}

\begin{document}

%%%%%%%%%%%%%%%%%%%%%%%%%%%%%%%%%%%%%%%%%%%%%%%%%%%%%%%%%%%%%%%%%%%%%%%
%%	title page information
%%%%%%%%%%%%%%%%%%%%%%%%%%%%%%%%%%%%%%%%%%%%%%%%%%%%%%%%%%%%%%%%%%%%%%%
\title[beamer manual]										%% optional, use only with long paper titles
{This is the 2021 MPIDR beamer template. \\ Beware of the rough edges! }

\author[Theile]{Tom Theile\inst{1}}

\institute[MPIDR]{\inst{1}Software Developer at the \textit{DCoDe Lab}}

%\Email{theile@mpidr.de}

\begin{frame}[mpidrbackground=1]
  \titlepage
\end{frame}



%%%%%%%%%%%%%%%%%%%%%%%%%%%%%%%%%%%%%%%%%%%%%%%%%%%%%%%%%%%%%%%%%%%%%%%
%%	section 1
%%%%%%%%%%%%%%%%%%%%%%%%%%%%%%%%%%%%%%%%%%%%%%%%%%%%%%%%%%%%%%%%%%%%%%%
\section{How the template works}

\begin{frame}[mpidrbackground=2]
  \tableofcontents
\end{frame}


\begin{frame}[mpidrbackground=2]
    \frametitle{How it works}
  Every slide has a background which is a pdf-export of the official MPIDR-Powerpoint template.\\
  
  You can choose the background by specifying  mpidrbackground={1...5}
  
  where {1...5} is a number between 1 and 5 (duuh!)
  
\end{frame} 


%%%%%%%%%%%%%%%%%%%%%%%%%%%%%%%%%%%%%%%%%%%%%%%%%%%%%%%%%%%%%%%%%%%%%%%
%%	subsection 1.1
%%%%%%%%%%%%%%%%%%%%%%%%%%%%%%%%%%%%%%%%%%%%%%%%%%%%%%%%%%%%%%%%%%%%%%%
%%	- subsection names are only displayed in the TOC or in the headline when using "Navigationon"
\subsection{How to use the template}


\begin{frame}[mpidrbackground=1, fragile,label=notleM2bis] %these options are only necessary for "verbatim" (showing latex code).
    \frametitle{Different Text Colors}
        %\setbeamercolor{item}{fg=green}
    \setbeamercolor{normal text}{fg=white}
    \usebeamercolor[fg]{normal text}
    
    If you use the slide with the green background, you have to change the text color to white.
    \begin{verbatim}
        \setbeamercolor{normal text}{fg=white}
        \usebeamercolor[fg]{normal text}
    \end{verbatim}
    
\end{frame}


%%%%%%%%%%%%%%%%%%%%%%%%%%%%%%%%%%%%%%%%%%%%%%%%%%%%%%%%%%%%%%%%%%%%%%%
%%	subection 2.1
%%%%%%%%%%%%%%%%%%%%%%%%%%%%%%%%%%%%%%%%%%%%%%%%%%%%%%%%%%%%%%%%%%%%%%%
\subsection{Using the MPIDR style}
\begin{frame}[mpidrbackground=2, fragile]%, because otherwise there would be some issues using "verbatim"
    \frametitle{Style Files}
  Any style file that you need for this document can be found in the subfolder \texttt{./mpidrtheme}.\\
  
  The style file is included by
	\begin{verbatim}
		\usepackage[mpidr]{./mpidr/mpidr21}
	\end{verbatim}
	in the preamble.
	
	You can edit and delete the footer (footline) in the mpidrtheme/mpidr21.sty-file.
	
	%\begin{verbatim}
	%	\usepackage[NavigationOn]{./mpidr/beamerouterthemeMPIDR}.
	%\end{verbatim}
\end{frame}


%%%%%%%%%%%%%%%%%%%%%%%%%%%%%%%%%%%%%%%%%%%%%%%%%%%%%%%%%%%%%%%%%%%%%%%
%%	section 1
%%%%%%%%%%%%%%%%%%%%%%%%%%%%%%%%%%%%%%%%%%%%%%%%%%%%%%%%%%%%%%%%%%%%%%%
\section{Other example stuff}



\begin{frame}[mpidrbackground=4]
    \frametitle{Slide with a lot of space}
 You can put a lot of stuff here...
\end{frame}

\begin{frame}[mpidrbackground=3]
    \frametitle{The MPIDR}
  This slide has a nice photo \\ of the Institute
\end{frame}
\begin{frame}[mpidrbackground=5]
    \frametitle{Swirly Background}
  The grey background is a little bit different this time
\end{frame}




%%%%%%%%%%%%%%%%%%%%%%%%%%%%%%%%%%%%%%%%%%%%%%%%%%%%%%%%%%%%%%%%%%%%%%%
%%	subsection 1.1
%%%%%%%%%%%%%%%%%%%%%%%%%%%%%%%%%%%%%%%%%%%%%%%%%%%%%%%%%%%%%%%%%%%%%%%
%%	- subsection names are only displayed in the TOC or in the headline when using "Navigationon"
\subsection{Largest Prime Number?}
\begin{frame}[mpidrbackground=2]

%%	frame titles are always displayed at the top of the frames
  \frametitle{There Is No Largest Prime Number}
  \begin{theorem}
		There is no largest prime number.
  \end{theorem}
  \begin{proof}
    \begin{enumerate}
    \item<1-| alert@1> Suppose $p$ were the largest prime number.
    \item<2-> Let $q$ be the product of the first $p$ numbers.
    \item<3-> Then $q+1$ is not divisible by any of them.
    \item<1-> Thus $q+1$ is also prime and greater than $p$.\qedhere
    \end{enumerate}
  \end{proof}
\end{frame}


%%%%%%%%%%%%%%%%%%%%%%%%%%%%%%%%%%%%%%%%%%%%%%%%%%%%%%%%%%%%%%%%%%%%%%%
%%	subsection 1.2
%%%%%%%%%%%%%%%%%%%%%%%%%%%%%%%%%%%%%%%%%%%%%%%%%%%%%%%%%%%%%%%%%%%%%%%
\subsection{Math mode}
%\mpidrbackground{1}
\begin{frame}[mpidrbackground=2]
    \frametitle{Formulae}
	Of course, there is no limit in typesetting mathematical formulae:
	\begin{align}
		V_{ab}^s\left(q, i z_\mu\right)
			&= \frac{V_{ab}(q)}{1-\sum_{c}V_{cc}(q)\Pi_{cc}\left(q, i z_\mu\right)} 
				\equiv \frac{V_{ab}(q)}{\varepsilon\left(q, i z_\mu\right)}  \\
		\rule{0pt}{2em}\varepsilon\left(q, iz_\mu\right) 
			&= 1-\sum_c V_{cc}(q)\Pi_{cc}\left(q, i z_\mu\right)
	\end{align}
	As you can see above, there is no need for using boxes.
\end{frame}


%%%%%%%%%%%%%%%%%%%%%%%%%%%%%%%%%%%%%%%%%%%%%%%%%%%%%%%%%%%%%%%%%%%%%%%
%%	subsection 1.3
%%%%%%%%%%%%%%%%%%%%%%%%%%%%%%%%%%%%%%%%%%%%%%%%%%%%%%%%%%%%%%%%%%%%%%%
\subsection{Lists}
\begin{frame}[mpidrbackground=2]
    \frametitle{Columns, Itemizing, numeration}
   This frame contains two columns for demonstrating bulleted and numbered lists. Font sizes will be changed automatically.
	
   \begin{columns}
      \column{0.45\textwidth}
         \begin{itemize}
            \item first item
                  \begin{itemize}
                     \item first nested item 
                     \item second nested item
                           \begin{itemize}
                              \item[--] first double-nested item
                              \item second double-nested item
                           \end{itemize}
                     \item third nested item
                  \end{itemize}
            \item second item
         \end{itemize}

      \column{0.45\textwidth}
         \begin{enumerate}
            \item Enumerations
                  \begin{enumerate}
                     \item may be
                           \begin{enumerate}
                              \item changed
                              \item in the
                           \end{enumerate}
                     \item same way.
                  \end{enumerate}
         \end{enumerate}
   \end{columns}

\end{frame}


%\interframe{Interframe.}


%%%%%%%%%%%%%%%%%%%%%%%%%%%%%%%%%%%%%%%%%%%%%%%%%%%%%%%%%%%%%%%%%%%%%%%
%%	subsection 1.4
%%%%%%%%%%%%%%%%%%%%%%%%%%%%%%%%%%%%%%%%%%%%%%%%%%%%%%%%%%%%%%%%%%%%%%%
\subsection{Not implemented yet} 

\begin{frame}[mpidrbackground=4] % Dieses Frame wird je nach Platz/Inhalt/Fuellstand automatisch geteilt
    \frametitle{Theorems, Proofs, other stuff are not \\ implemented yet}
   \begin{theorem}
      The color scheme for this theorem-box is used automatically {\upshape(\texttt{./mpidr/beamercolorthemeMPIDR.sty})}.
   \end{theorem}

   \begin{proof}
      Proof-boxes always end with a special symbol, a small blue square.
   \end{proof}

   \begin{block}{Example :)}
      The title of this block is ``\insertblocktitle''.
   \end{block}

   \begin{alertblock}{Alert block}
      An alert block.
   \end{alertblock}

   %\begin{exampleblock}{Example block}
    %  An example block.
   %\end{exampleblock}

\end{frame}







\end{document}